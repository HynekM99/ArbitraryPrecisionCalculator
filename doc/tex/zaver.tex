Vytvořená kalkulačka dokáže provádět všechny požadované operace, často i za relativně krátkou dobu běhu. Výrazy v infixové formě zpracovat umí a během testovaní nebyl nalezen žádný výraz, který by se vyhodnotil nesprávně. Speciální případ výrazu s operátorem mocniny, po kterém následuje operátor negace, je ošetřen, správnost tohoto ošetření nemusí být ale jednoznačná pro některé složitější výrazy. Implementovány jsou jak režim interaktivní, tak režim zpracování souboru, včetně chybových hlášení nepodporovaných příkazů, či nedefinovaných matematických operací. `Nekonečně velká' čísla sice kalkulačka zpracovat neumí, čísla ale mohou být tak velká, že je lze označit za `prakticky nekonečně velká'.

Jedním z problému k vyřešení byla příliš dlouhá doba kontroly programem \verb|valgrind|, která byla  způsobena zbytečnými dynamickými alokacemi a realokacemi struktury \verb|mpt| a bylo nutné celý program přepsat tak, aby se struktury alokovaly na zásobníku. 
Rychlost vyhodnocování některých výrazů může u větších čísel být relativně malá a určitě existují algoritmy, které by byly pro provedení vybraných matematických operací lepší. Právě takové algoritmy pravděpodobně využívá např.~knihovna \href{https://gmplib.org/}{\textit{GMP}}, která je lepší alternativou našeho řešení.